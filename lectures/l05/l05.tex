\input{../common}

\usepackage{tikz}
\usetikzlibrary{positioning}

\title{Programming Techniques --- Pointers and Derived Types}
\subtitle{Lecture: Pointers, Dynamic Data, and Derived Types}
\author{Hannu Parviainen}
\date{\today}

\begin{document}

\begin{frame}
  \titlepage
\end{frame}

\section{Allocatable arrays}

\begin{frame}
	\centering
	\Huge \textbf{Dynamic Memory Allocation}
\end{frame}


\begin{frame}[fragile]{Static memory allocation}
  \begin{columns}[T]
    \column{0.48\textwidth}
    \begin{block}{Concept}
      \begin{itemize}
        \item Array size is fixed at compile time.
        \item Memory allocated once when the program starts.
        \item Lifetime = entire execution of the program (or scope).
        \item Efficient, but inflexible when array size is not known in advance.
      \end{itemize}
    \end{block}
    \column{0.52\textwidth}
    \lstinputlisting[language=Fortran]{ex01.f90}
  \end{columns}
\end{frame}


\begin{frame}[fragile]{Dynamic memory allocation}
  \begin{columns}[T]
    \column{0.48\textwidth}
    \begin{block}{Concept}
      \begin{itemize}
        \item Size and shape chosen at runtime.
        \item Memory allocated with \texttt{allocate}, released with \texttt{deallocate}.
        \item Lifetime = until explicitly deallocated.
        \item More flexible, but requires careful management.
      \end{itemize}
    \end{block}
    \column{0.52\textwidth}
    \lstinputlisting[language=Fortran]{ex02.f90}
  \end{columns}
\end{frame}


\begin{frame}[fragile]{Allocatable Arrays in Fortran}
\begin{block}{Declaration}
\begin{lstlisting}[language=Fortran]
integer, dimension(:),   allocatable :: ages
real,    dimension(:,:), allocatable :: speed
\end{lstlisting}
 \end{block}
 \begin{block}{Allocation}
\begin{lstlisting}[language=Fortran]
integer :: isize, ierr
read *, isize

allocate(ages(isize), stat=ierr)
if (ierr /= 0) print *, 'ages: allocation failed'

allocate(speed(0:isize-1, 10), stat=ierr)
if (ierr /= 0) print *, 'speed: allocation failed'
\end{lstlisting}
 \end{block}
 \begin{block}{Note}
   \texttt{stat} is optional but recommended to detect errors.
 \end{block}
\end{frame}


\begin{frame}[fragile]{Deallocation and Status}
 \begin{block}{Deallocate}
\begin{lstlisting}[language=Fortran]
integer :: ierr
if (allocated(ages)) deallocate(ages, stat=ierr)
if (allocated(speed)) deallocate(speed, stat=ierr)
\end{lstlisting}
 \end{block}
 \begin{block}{Guidelines}
   \begin{itemize}
     \item Only deallocate allocatable arrays that are currently allocated.
     \item Prefer checking \texttt{allocated(array)} before deallocation.
     \item In procedures, deallocate before exit.
   \end{itemize}
 \end{block}
\end{frame}



\section{Pointers}

\begin{frame}
	\centering
	\Huge \textbf{Pointers}
\end{frame}

\begin{frame}{Fortran Pointers: Introduction}
  \begin{block}{What is a pointer?}
    \begin{itemize}
      \item A \textbf{pointer} is a variable that does not store a value directly,
            but instead \textbf{points to a memory location}.
      \item In Fortran, pointers can be associated with
            \begin{itemize}
              \item other variables,
              \item array sections,
              \item dynamically allocated memory.
            \end{itemize}
    \end{itemize}
  \end{block}

  \begin{block}{Why are they useful?}
    \begin{itemize}
      \item Enable \textbf{dynamic memory management}
            (allocate and free memory at runtime).
      \item Allow multiple variables to \textbf{share the same data}.
      \item Needed for building \textbf{linked data structures}
            (lists, trees, graphs).
      \item Provide flexibility similar to references in modern languages.
    \end{itemize}
  \end{block}
\end{frame}


\begin{frame}{Python vs.~Fortran Variables}
  \begin{block}{Key difference}
    \begin{itemize}
      \item \textbf{Python variables} are \emph{references} to objects in memory.
        \begin{itemize}
          \item Assigning one variable to another does not copy the object.
          \item Example: \texttt{b = a} makes \texttt{b} point to the same object as \texttt{a}.
        \end{itemize}
      \item \textbf{Fortran variables} are normally \emph{values} stored directly.
        \begin{itemize}
          \item Assigning one variable to another copies the value.
          \item No implicit references like in Python.
        \end{itemize}
    \end{itemize}
  \end{block}

  \textbf{Python variables act like Fortran pointers by default}, while plain Fortran variables are independent copies.

\end{frame}


\begin{frame}[fragile]{Pointer Declaration}
  \begin{block}{Definition}
    \begin{itemize}
      \item A variable with the \texttt{pointer} attribute.
      \item Has static type, kind, and rank determined by its declaration.
    \end{itemize}
  \end{block}

  \begin{block}{Examples}
\begin{lstlisting}[language=Fortran]
real, pointer :: ptor
real, dimension(:,:), pointer :: ptoa
\end{lstlisting}
    \begin{itemize}
      \item \texttt{ptor} is a pointer to a scalar real target.
      \item \texttt{ptoa} is a pointer to a 2D array of reals.
    \end{itemize}
  \end{block}
\end{frame}


\begin{frame}[fragile]{Target Declaration}

  \begin{block}{Targets must have the \texttt{target} attribute}
\begin{lstlisting}[language=Fortran]
real, target :: x, y
real, dimension(5,3), target :: a, b
real, dimension(3,5), target :: c
\end{lstlisting}
    \begin{itemize}
      \item \texttt{x} or \texttt{y} may become associated with \texttt{ptor}.
      \item \texttt{a}, \texttt{b}, or \texttt{c} may become associated with \texttt{ptoa}.
    \end{itemize}
  \end{block}
\end{frame}

\begin{frame}{Pointer Manipulation}
  \begin{block}{Operators}
    \begin{itemize}
      \item \texttt{=>} pointer assignment: alias pointer with a target.
      \item \texttt{=} normal assignment: assign value to space pointed at.
    \end{itemize}
    Pointer assignment makes the pointer and target reference the same space, while normal assignment changes the value.
  \end{block}
\end{frame}

\begin{frame}[fragile]{Pointer Assignment}
\begin{lstlisting}[language=Fortran]
real, target :: x, y
real, pointer :: ptor

x = 3.14159
ptor => y
ptor = x
\end{lstlisting}
  \begin{itemize}
    \item \texttt{x} and \texttt{ptor} have the same value.
    \item \texttt{ptor} is an alias for \texttt{y}, so the assignment sets \texttt{y = 3.14159}.
    \item Changing \texttt{x} later does not change \texttt{ptor} or \texttt{y}.
  \end{itemize}
\end{frame}

\begin{frame}[fragile]{Dynamic Targets}

  \begin{block}{Targets can be created dynamically by allocation}
\begin{lstlisting}[language=Fortran]
allocate(ptor, stat=ierr)
allocate(ptoa(n*n, 2*k-1), stat=ierr)
\end{lstlisting}
    \begin{itemize}
      \item First: allocates a scalar real as target of \texttt{ptor}.
      \item Second: allocates a rank-2 real array as target of \texttt{ptoa}.
      \item Re-allocating an already associated pointer is not an error.
    \end{itemize}
  \end{block}
\end{frame}

\begin{frame}[fragile]{Association Status}
  \begin{block}{Test with \texttt{associated()}}
\begin{lstlisting}[language=Fortran]
associated(ptoa)
associated(ptoa, arr)
\end{lstlisting}
    \begin{itemize}
      \item Returns \texttt{.true.} if pointer is defined and associated.
      \item Second form tests association with a specific target.
    \end{itemize}
  \end{block}
\end{frame}

\begin{frame}[fragile]{Pointer Disassociation}
  \begin{block}{Nullification}
\begin{lstlisting}[language=Fortran]
nullify(ptor)
\end{lstlisting}
    \begin{itemize}
      \item Breaks pointer–target connection.
      \item Good practice to nullify before use.
    \end{itemize}
  \end{block}

  \begin{block}{Deallocation}
\begin{lstlisting}[language=Fortran]
deallocate(ptoa, stat=ierr)
\end{lstlisting}
    \begin{itemize}
      \item Breaks connection and deallocates target.
    \end{itemize}
  \end{block}
\end{frame}

\begin{frame}[fragile]{Practical Example}
\begin{lstlisting}[language=Fortran]
real, dimension(100,100), target :: app1, app2
real, dimension(:,:), pointer :: prev_app, next_app, swap

prev_app => app1
next_app => app2
prev_app = initial_app(...)

do
  next_app = iteration_function_of(prev_app)
  if (abs(maxval(next_app-prev_app))<0.0001) exit
  swap => prev_app
  prev_app => next_app
  next_app => swap
end do
\end{lstlisting}
  \begin{block}{Note}
    Pointers avoid copying large matrices in iterative algorithms.
  \end{block}
\end{frame}



\section{Derived Types}

\begin{frame}
	\centering
	\Huge \textbf{Derived Types}
\end{frame}

\begin{frame}[fragile]{Derived Types}

\begin{columns}[T]
  \column{0.5\textwidth}
  \begin{block}{Example: 3D vector type}
    \lstinputlisting[firstline=1,lastline=7]{ex_types.f90}
  \end{block}
  \begin{itemize}
  \item Basic Fortran types can be combined to create more complex \textit{derived} types.
  \item Derived type definitions should be placed in a module.
  \end{itemize}

  \column{0.5\textwidth}
  \begin{block}{Example: type usage}
  \lstinputlisting[firstline=9,lastline=15]{ex_types.f90}
  \end{block}
  \begin{itemize}
  \item Variables are declared using the \texttt{type} statement.
  \item Type values can be initialised in the same order as they are defined inside the type.
  \end{itemize}

\end{columns}
\end{frame}

\begin{frame}[fragile]{Supertypes}
  Previously defined types can be used as components of other derived types.
\begin{columns}[T,onlytextwidth]
  \column{0.5\textwidth}
  \begin{block}{Example: sphere}
    \lstinputlisting[firstline=1,lastline=10]{ex_types_2.f90}
  \end{block}

    \column{0.49\textwidth}
    \begin{block}{Example: sphere usage}
    \lstinputlisting[firstline=12,lastline=18,basicstyle=\ttfamily\scriptsize]{ex_types_2.f90}
  \end{block}
  \end{columns}
\end{frame}


% --- Derived Type Assignment ---
\begin{frame}[fragile]{Derived type assignment}
  \begin{columns}[T]
    \column{0.52\textwidth}
    \begin{block}{Ways to assign}
      \begin{itemize}
        \item Component-by-component using \texttt{\%}.
        \item As a whole object using a constructor.
        \item Assignment between two objects of the same derived type is intrinsic.
      \end{itemize}
      The derived-type component of a composite (e.g., \texttt{sphere\%centre}) must also be set via a constructor.
    \end{block}

    \column{0.48\textwidth}
\begin{lstlisting}[language=Fortran]
! component by component
pt1%x = 1.0
p%radius = 3.0
p%centre%x = 1.0

! whole-object with constructors
pt1 = vector3d(1., 2., 3.)
p%centre = vector3d(1., 2., 3.)
p = sphere(p%centre, 10.)
p = sphere(vector3d(1.,2.,3.), 10.)

! intrinsic assignment between objects
ball = p
\end{lstlisting}
  \end{columns}
\end{frame}

\begin{frame}[fragile]{Derived type I/O}
  \begin{columns}[T]
    \column{0.52\textwidth}
    \begin{block}{Unformatted I/O}
      \begin{itemize}
        \item If a derived type has no pointer or private components, it can be printed/read “normally.”
        \item I/O proceeds component-by-component in order.
      \end{itemize}
      Example equivalence shown at right.
    \end{block}

    \column{0.48\textwidth}
\begin{lstlisting}[language=Fortran]
print *, p

print *, p%centre%x, p%centre%y, &
         p%centre%z, p%radius
\end{lstlisting}
  \end{columns}
\end{frame}

\begin{frame}[fragile]{Pointer components of derived types}
  \begin{columns}[T]
    \column{0.52\textwidth}
    \begin{block}{Allocatable vs pointer}
      \begin{itemize}
        \item \texttt{allocatable} arrays cannot be components in a derived type; \texttt{pointer} components can.
        \item Enables dynamic-size structures such as growable strings.
      \end{itemize}
      Example: a vector-of-characters pointer for a simple string type.
    \end{block}

    \column{0.48\textwidth}
\begin{lstlisting}[language=Fortran]
type vstring
  character, dimension(:), pointer :: chars
end type vstring

type(vstring) :: pvs1
! ...
allocate(pvs1%chars(5))
pvs1%chars = (/"H","e","l","l","o"/)
! prints: H e l l o
\end{lstlisting}
  \end{columns}
\end{frame}


% \begin{frame}[fragile]{Pointers and recursive data structures}
%   \begin{columns}[T]
%     \column{0.52\textwidth}
%     \begin{block}{Linked lists via pointers}
%       \begin{itemize}
%         \item Derived types with pointer components support lists/trees.
%         \item Note subtle assignment semantics with pointer components:
%       \end{itemize}
% \begin{lstlisting}[language=Fortran]
% type(cell)        :: a
% type(cell),target :: b
% a = b
% ! equivalent to:
% ! a%val  = b%val
% ! a%next => b%next
% \end{lstlisting}
%       The pointer component is pointer-assigned (\texttt{=>}), not value-copied.
%     \end{block}
%
%     \column{0.48\textwidth}
% \begin{lstlisting}[language=Fortran]
% type cell
%   integer :: val
%   type(cell), pointer :: next
% end type cell
% \end{lstlisting}
%   \end{columns}
% \end{frame}
%
% \begin{frame}[fragile]{Linked list: construction}
%   \begin{columns}[T]
%     \column{0.52\textwidth}
%     \begin{block}{Idea}
%       Build a list with a \texttt{head} node, appending values read from input until termination. Pointers avoid copying and give O(1) append by keeping a tail pointer.
%     \end{block}
%
%     \column{0.48\textwidth}
% \begin{lstlisting}[language=Fortran]
% program thingy
%   implicit none
%   type(cell), target  :: head
%   type(cell), pointer :: curr, temp
%   integer :: k
%
%   head%val = 0
%   nullify(head%next)
%   curr => head
%   do
%     read*, k
%     allocate(temp)
%     temp%val = k
%     nullify(temp%next)
%     curr%next => temp
%     curr => temp
%   end do
% end program thingy
% \end{lstlisting}
%   \end{columns}
% \end{frame}
%
% \begin{frame}[fragile]{Linked list: traversal}
%   \begin{columns}[T]
%     \column{0.52\textwidth}
%     \begin{block}{Walk-through}
%       Traverse from \texttt{head} to the end by following \texttt{next}. Stop when \texttt{next} is disassociated.
%     \end{block}
%
%     \column{0.48\textwidth}
% \begin{lstlisting}[language=Fortran]
% curr => head
% do
%   print *, curr%val
%   if (.not. associated(curr%next)) exit
%   curr => curr%next
% end do
% \end{lstlisting}
%   \end{columns}
% \end{frame}



\begin{frame}[fragile]{Arrays of pointers}
  \begin{columns}[T]
    \column{0.52\textwidth}
    \begin{block}{Notes}
      \begin{itemize}
        \item You can make arrays whose elements are themselves pointers.
        \item You cannot reference a whole array of pointer components in one go.
        \item If desired, the array-of-pointers container could be \texttt{allocatable}.
      \end{itemize}
      See valid vs. invalid examples at right.
    \end{block}

    \column{0.48\textwidth}
\begin{lstlisting}[language=Fortran]
type iptr
  integer, pointer :: compon
end type iptr

type(iptr), dimension(100) :: ints

! ok:
ints(10)%compon

! not ok (whole array component):
! ints(:)%compon
\end{lstlisting}
  \end{columns}
\end{frame}


\section{Custom Operators}

\begin{frame}
	\centering
	\Huge \textbf{Custom Operators}
\end{frame}

\begin{frame}[fragile]
  \frametitle{Custom Operators}
  \begin{columns}[T]
    \begin{column}{0.5\textwidth}
      \begin{itemize}
        \item Fortran allows you to define complex derived types.
        \item But using them directly for calculations leads to cumbersome code (and bugs, lots and lots of bugs)
      \end{itemize}
    \end{column}

    \begin{column}{0.5\textwidth}
    \lstinputlisting[firstline=1,lastline=6]{code.f90}
    \lstinputlisting[firstline=8,lastline=16]{code.f90}
    \end{column}
  \end{columns}
\end{frame}

\begin{frame}[fragile]
  \frametitle{Custom Operators}
   Code can be simplified (and made more robust) by using functions.
  \begin{columns}[T]
    \begin{column}{0.5\textwidth}
          \lstinputlisting[firstline=18,lastline=35]{code.f90}
    \end{column}

    \begin{column}{0.5\textwidth}
    \lstinputlisting[firstline=37,lastline=45]{code.f90}
    \end{column}
  \end{columns}
\end{frame}

\begin{frame}[fragile]
  \frametitle{Custom Operators}
   And defining custom operators can make your code even cleaner.
  \begin{columns}[T]
    \begin{column}{0.5\textwidth}
          \lstinputlisting[firstline=48,lastline=69,basicstyle=\ttfamily\tiny]{code.f90}
    \end{column}

    \begin{column}{0.5\textwidth}
    \lstinputlisting[firstline=75,lastline=83,basicstyle=\ttfamily\tiny]{code.f90}
    \end{column}
  \end{columns}
\end{frame}


\begin{frame}[fragile]
  \frametitle{How to Define a Custom Operator}
  Custom operators are defined inside a module using the \texttt{operator} keyword.
  \vspace*{2mm}

  \begin{columns}[T]
    \begin{column}{0.5\textwidth}
      \begin{itemize}
            \item Standard operators (+, -, *, /) \lstinputlisting[firstline=86,lastline=88]{code.f90}
      \end{itemize}
    \end{column}

    \begin{column}{0.5\textwidth}
      \begin{itemize}
            \item A new operator named \texttt{.op.} \lstinputlisting[firstline=90,lastline=92]{code.f90}
      \end{itemize}    \end{column}
  \end{columns}

  \vspace*{2mm}
  The functions implementing the operators are defined later in the module after the \texttt{contains} keyword.
\end{frame}



\end{document}
